% ------------------------------------------------------------
\subsection{Choice rule}
% ------------------------------------------------------------
\begin{frame}[fragile]{Choice rule}
  \label{eqn:choice:rule}
  \begin{itemize}
  \item \structure{Idea} Choices over subsets of literals
  \item \structure{Syntax} A \alert{choice rule} is of the form
    \[
    \{a_1,\dots,a_m\}\leftarrow a_{m+1},\dots,a_n,\neg a_{n+1},\dots,\neg a_o
    \]
    where $0\leq m\leq n\leq o$ and each $a_i$ is an atom for $1\leq i\leq o$
    \smallskip
  \item<2-> \structure{Informal meaning} If the body is satisfied by the stable model at hand,\\
    then any subset of $\{a_1,\dots,a_m\}$ can be included in the stable model
    \medskip
  \item<3-> \structure{Example}  % ,belowskip=2pt,aboveskip=2pt
\begin{lstlisting}[basicstyle=\ttfamily\small]
{ buy(pizza);buy(wine);buy(corn) } :- at(grocery).
\end{lstlisting}
  \item<4-> \structure{Example program}
    \begin{align*}
      & \{\;\{a\}\leftarrow b,\ b\leftarrow\;\}
    \end{align*}
  \end{itemize}
\end{frame}
% ------------------------------------------------------------
\begin{frame}{Embedding in normal rules}
  \begin{itemize}
  \item A choice rule of form
    \[
    \{\alert<3>{a_1},\dots,\alert<3>{a_m}\}\leftarrow \alert<2>{a_{m+1},\dots,a_n,\neg a_{n+1},\dots,\neg a_o}
    \]
    can be translated into $2m+1$ normal rules
    \[
    \begin{array}{rcl@{\qquad}c@{\qquad}rcl}
      b&\leftarrow&\multicolumn{5}{l}{\alert<2>{a_{m+1},\dots,a_n, \neg a_{n+1},\dots,\neg a_o}}
      \\[5pt]
      \alert<3>{a_1}&\leftarrow&b,\neg a'_1&\dots&\alert<3>{a_m}&\leftarrow&b,\neg a'_m\\
               a'_1 &\leftarrow&  \neg a_1 &\dots&         a'_m &\leftarrow&  \neg a_m
    \end{array}
    \]
    by introducing new atoms $b,a'_1,\dots,a'_m$
  \end{itemize}
\end{frame}
% ------------------------------------------------------------
\subsection{Cardinality rule}
% ------------------------------------------------------------
\begin{frame}[fragile]{Cardinality rule}
  \label{eqn:cardinality:rule}
  \begin{itemize}
  \item \structure{Idea}
    Control (lower) cardinality of subsets of literals
  \item \structure{Syntax}
    A \alert{cardinality rule} is the form
    \[
    a_0\leftarrow l~\{\ a_1,\dots,a_m,\neg a_{m+1},\dots,\neg a_n\ \}
    \]
    where $0\leq m\leq n$ and each $a_i$ is an atom for $1\leq i\leq n$;\\
    $l$ is a non-negative integer (acting as a \alert{lower bound} on the body)
    \smallskip
  \item<2-> \structure{Informal meaning}
    The head atom belongs to the stable model,\\
    if at least $l$ positive/negative body literals are in/excluded in the stable model
    \medskip
  \item<3-> \structure{Example}
\begin{lstlisting}[basicstyle=\ttfamily\small]
pass(c42) :- 2 { pass(a1); pass(a2); pass(a3) }.
\end{lstlisting}
  \item<4-> \structure{Example program}
    \begin{align*}
      & \{\;a\leftarrow 1\;\{b,c\},\ b\leftarrow\;\}
    \end{align*}
  \end{itemize}
\end{frame}
%------------------------------------------------------------
\begin{frame}{Embedding in normal rules}
\begin{itemize}
\item A cardinality rule of form
\[
\alert<2>{a_0}\leftarrow\alert<2>{l}~\{\ \alert<5>{a_1,\dots,a_m},\alert<6>{\neg a_{m+1},\dots,\neg a_n}\ \}
\]
is translated into the normal rule \
\(
\alert<2>{a_0}\leftarrow \mathit{ctr}(1,\alert<2>{l})
\)
\ \uncover<4->{and}
\item<only@4-> [] for $0\leq k\leq l$ the rules
\[
\begin{array}{rclp{20pt}l}
\mathit{ctr}(i, k{+}1)&\leftarrow&\mathit{ctr}({i+1}, k),\alert<5>{a_i}
&&\\
\mathit{ctr}(i, k)&\leftarrow&\mathit{ctr}({i+1}, k)
&&\text{for }\alert<5>{1}\leq i\leq \alert<5>{m}
\\[5pt]
\mathit{ctr}(j, k{+}1)&\leftarrow&\mathit{ctr}({j+1}, k),\alert<6>{\neg a_j}
\\
\mathit{ctr}(j, k)&\leftarrow&\mathit{ctr}({j+1}, k)
&&\text{for }\alert<6>{m+1}\leq j\leq \alert<6>{n}
\\[5pt]
\mathit{ctr}(\alert<7>{n+1}, \alert<7>{0})&\leftarrow&
&&
\end{array}
\]
\item<3->
The atom $\mathit{ctr}(i,j)$ represents the fact that at least $j$ of the
literals having an equal or greater index than $i$, are in a stable model
\end{itemize}
\end{frame}
% ------------------------------------------------------------
\begin{frame}{An example}
\begin{itemize}
\item Program
  \(
  \{  a\leftarrow,\ c\leftarrow 1~\{a,b\}  \}
  \)
  has the stable model $\{a,c\}$
\item<2-> Translating the cardinality rule yields the rules
\[
\begin{array}[t]{rcl}
a&\leftarrow&
\end{array}
\qquad
\begin{array}[t]{rcl}
c&\leftarrow& \mathit{ctr}(1,1)
\\
\mathit{ctr}(1, 2)&\leftarrow&\mathit{ctr}(2, 1),a
\\
\mathit{ctr}(1, 1)&\leftarrow&\mathit{ctr}(2, 1)
\\
\mathit{ctr}(2, 2)&\leftarrow&\mathit{ctr}(3, 1),b
\\
\mathit{ctr}(2, 1)&\leftarrow&\mathit{ctr}(3, 1)
\\
\mathit{ctr}(1, 1)&\leftarrow&\mathit{ctr}(2, 0),a
\\
\mathit{ctr}(1, 0)&\leftarrow&\mathit{ctr}(2, 0)
\\
\mathit{ctr}(2, 1)&\leftarrow&\mathit{ctr}(3, 0),b
\\
\mathit{ctr}(2, 0)&\leftarrow&\mathit{ctr}(3, 0)
\\
\mathit{ctr}(3, 0)&\leftarrow&
\end{array}
\]
having stable model
\(
\{
a,
\mathit{ctr}(3, 0),
\mathit{ctr}(2, 0),
\mathit{ctr}(1, 0),
\mathit{ctr}(1, 1),
c
\}
\)
\end{itemize}
\end{frame}
% ------------------------------------------------------------
\begin{frame}{\dots\ and vice versa}
  \begin{itemize}
  \item A normal rule
    \[
    a_0\leftarrow a_1,\dots,a_m, \neg a_{m+1},\dots,\neg a_n
    \]
  \item[] can be represented by the cardinality rule
    \[
    a_0\leftarrow n \ \{a_1,\dots,a_m, \neg a_{m+1},\dots,\neg a_n\}
    \]
  \end{itemize}
\end{frame}
% ------------------------------------------------------------
\begin{frame}{Cardinality rules with upper bounds}
  \begin{itemize}
  \item A rule of the form
    \begin{equation}\label{eq:cardinality:constraint}
      a_0\leftarrow \alert<3>{l~\{\ a_1,\dots,a_m,\neg a_{m+1},\dots,\neg a_n\ \}~u}
    \end{equation}
    where $0\leq m\leq n$ and each $a_i$ is an atom for $1\leq i\leq n$;\\
    $l$ and $u$ are non-negative integers
  \item<2-> []
    stands for
    \[
    \begin{array}{rcl}
      a_0 &\leftarrow& b, \neg c\\
      b   &\leftarrow& l~\{\ a_1,\dots,a_m,\neg a_{m+1},\dots,\neg a_n\ \}\\
      c   &\leftarrow& u{+}1~\{\ a_1,\dots,a_m,\neg a_{m+1},\dots,\neg a_n\ \}
    \end{array}
    \]
    where $b$ and $c$ are new symbols
    \medskip
  \item<3-> \structure{Note}
    The expression in the body of the cardinality rule \eqref{eq:cardinality:constraint} is
    referred to as a \alert<3>{cardinality constraint} with lower and upper bound $l$ and $u$
  \end{itemize}
\end{frame}
% ------------------------------------------------------------
\begin{frame}{Cardinality constraints}
  \begin{itemize}
  \item \structure{Syntax} A \alert{cardinality constraint} is of the form
    \[
    l~\{\ a_1,\dots,a_m,\neg a_{m+1},\dots,\neg a_n\ \}~u
    \]
    where $0\leq m\leq n$ and each $a_i$ is an atom for $1\leq i\leq n$;\\
    $l$ and $u$ are non-negative integers
    \smallskip
  \item<2-> \structure{Informal meaning} A cardinality constraint is satisfied by a stable model $X$,
    if the number of its contained literals satisfied by $X$ is between $l$ and $u$ (inclusive)
  \item<3->
    In other words, if
    \[
    l\leq|\left(\{a_1,\dots,a_m\}\cap X\right)\cup\left(\{a_{m+1},\dots,a_n\}\setminus X\right)|\leq u
    \]
  % \item \structure{Conditions} \qquad
  %   \(
  %   l\ \{a_1:b_1,\dots,a_m:b_m\}\ u
  %   \)
  %
  %   where $B_1,\dots,B_m$ are used for restricting instantiations of
  %   variables occurring in $a_1,\dots,a_m$
  \end{itemize}
\end{frame}
% ------------------------------------------------------------
\begin{frame}[fragile]{Cardinality constraints as heads}
  \begin{itemize}
  \item A rule of the form
    \[
    l~\{a_1,\dots,a_m,\neg a_{m+1},\dots,\neg a_n\}~u\leftarrow a_{n+1},\dots,a_o,\neg a_{o+1},\dots,\neg a_p
    \]
    where $0\leq m\leq n\leq o\leq p$ and each $a_i$ is an atom for $1\leq i\leq p$;\\
    $l$ and $u$ are non-negative integers
    \uncover<2->{stands for
      \[
    \begin{array}{rcl}
      b                 &\leftarrow& a_{n+1},\dots,a_o, \neg a_{o+1},\dots,\neg a_p\\
      \{a_1,\dots,a_m\} &\leftarrow& b\\
      c                 &\leftarrow& l~\{a_1,\dots,a_m,,\neg a_{m+1},\dots,\neg a_n\}~u\\
      &\leftarrow& b, \neg c
    \end{array}
    \]
    where $b$ and $c$ are new symbols}
    \smallskip
  \item<3-> \structure{Example}
\begin{lstlisting}[basicstyle=\ttfamily\small]
1{color(v42,red);color(v42,green);color(v42,blue)}1.
\end{lstlisting}
%    \lstinline[basicstyle=\ttfamily\small]{1 \{ color(v42,red);color(v42,green);color(v42,blue) \} 1.}
  \end{itemize}
\end{frame}
% ------------------------------------------------------------
\begin{frame}{Full-fledged cardinality rules}
  \begin{itemize}
  \item A rule of the form
    \[
    \alert<6>{l_0~S_0~u_0}\leftarrow\alert<4>{l_1~S_1~u_1},\dots,\alert<4>{l_n~S_n~u_n}
    \]
    where each $\alert<3>{l_i~S_i~u_i}$ is a cardinality constraint for $0\leq i\leq n$
  \item<2-> [] stands for % $0\leq i\leq n$
    \[
    \begin{array}{rclp{20pt}rcl}
            \alert<5>{a}&\alert<5>{\leftarrow}&\multicolumn{5}{l}{\alert<4,5>{b_1},\dots,\alert<4,5>{b_n},\alert<4,5>{\neg
                                                c_1},\dots,\alert<4,5>{\neg c_n}}\\[10pt]
\alert<6>{\poslits{S_0}}&\alert<6>{\leftarrow}&\alert<5>{a}\\
                        &\alert<6>{\leftarrow}&\alert<5>{a}\alert<6>{,\neg b_0}&&\alert<3-4,6>{b_i}&\alert<3-4,6>{\leftarrow}&\alert<3-4,6>{l_i~S_i}\\
                        &\alert<6>{\leftarrow}&\alert<5>{a}\alert<6>{,     c_0 }&&\alert<3-4,6>{c_i}&\alert<3-4,6>{\leftarrow}&\alert<3-4,6>{u_i{+}1~S_i}
    \end{array}
    \]
    where $a,b_i,c_i$ are new symbols (and $\cdot^+$ is defined as on Slide~\ref{eqn:rule})
  \end{itemize}
\end{frame}
% ------------------------------------------------------------
\subsection{Weight rule}
% ------------------------------------------------------------
\begin{frame}{Weight rule}
  \label{eqn:weight:rule}
  \begin{itemize}
  \item \structure{Idea} Bound (lower) sum of subsets of literal weights
  \item \structure{Syntax} A \alert{weighted literal} $w:k$ associates the weight $w$ with literal $k$
  \item \structure{Syntax} A \alert{weight rule} is the form
    \[
      a_0\leftarrow l~\{\ w_1:a_1,\dots,w_m:a_m,w_{m+1}:\neg a_{m+1},\dots,w_n:\neg a_n\ \}
    \]
    where $0\leq m\leq n$ and each $a_i$ is an atom;\\
    $l$ and $w_i$ are integers for $1\leq i\leq n$
    \medskip
  \item<2-> \structure{Informal meaning}
    The head atom belongs to the stable model,\\
    if the sum of weights associated with positive/negative body literals in/excluded in the stable model
    is at least $l$
    \medskip
  \item<3-> \structure{Note} A cardinality rule is a weight rule where $w_i=1$  for $0\leq i\leq n$
  \end{itemize}
\end{frame}
% ------------------------------------------------------------
\begin{frame}[fragile]{Weight constraints}
  \begin{itemize}
  \item \structure{Syntax} A \alert{weight constraint} is of the form
    \[
    l~\{\ w_1:a_1,\dots,w_m:a_m,w_{m+1}:\neg a_{m+1},\dots,w_n:\neg a_n\ \}~u
    \]
    where $0\leq m\leq n$ and each $a_i$ is an atom;\\
    $l,u$ and $w_i$ are integers for $1\leq i\leq n$
    \smallskip
  \item<2-> \structure{Meaning} A weight constraint is satisfied by a stable model $X$, if
    \[
    l\leq\left(
      \textstyle{\sum_{1\leq i\leq m, a_i    \in X}}\; w_i
      +
      \textstyle{\sum_{m <   i\leq n, a_i\not\in X}}\; w_i
    \right)\leq u
    \]
  \item<3-> \structure{Note} (Cardinality and) weight constraints amount to constraints on (count and) sum aggregate functions
    \medskip
  \item<4-> \structure{Example}
\begin{lstlisting}[basicstyle=\ttfamily\small]
5 { 4:course(db); 6:course(ai); 3:course(xml) } 10
\end{lstlisting}
  \end{itemize}
\end{frame}
% ----------------------------------------------------------------------
%
%%% Local Variables:
%%% mode: latex
%%% TeX-master: "../../main"
%%% End:
