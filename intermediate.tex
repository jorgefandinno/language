\section{Intermediate formats}
%------------------------------------------------------------
\subsection{smodels format}
%------------------------------------------------------------
\begin{frame}{\smodels\ format}
  \begin{itemize}
  \item The \smodels\ format consists of
    \begin{itemize}
    \item normal rules
    \item choice rules
    \item cardinality rules
    \item weight rules
    \item minimization statements
    \end{itemize}
  \item Block-oriented format
  \item<2-> \structure{Note} Minimization statements are not part of the logic program
  \end{itemize}
\end{frame}
%------------------------------------------------------------
\begin{frame}[c]{\smodels\ format in detail}
  \footnotesize
  \newcommand{\myspace}{\mbox{\textvisiblespace}}
  \begin{tabular}{|l|}
      \hline
      Type/Format\\
      \hline
      \hline
      Normal rule
      % \eqref{eqn:rule},
      Slide~\pageref{eqn:rule}
      \\
      \(
      1\myspace
      \iota(a_0)\myspace
      n\myspace
      n\!-\!m\myspace
      \iota(a_{m+1})\myspace\dots\myspace\iota(a_{n})\myspace
      \iota(a_1)\myspace\dots\myspace\iota(a_{m})
      \)\\
      \hline
      Cardinality rule
      % \eqref{eqn:cardinality:rule},
      Slide~\pageref{eqn:cardinality:rule}
      \\
      \(
      2\myspace
      \iota(a_0)\myspace
      n\myspace
      n\!-\!m\myspace
      l\myspace
      \iota(a_{m+1})\myspace\dots\myspace\iota(a_{n})\myspace
      \iota(a_1)\myspace\dots\myspace\iota(a_{m})
      \)\\
      \hline
      Choice rule
      % \eqref{eqn:choice:rule},
      Slide~\pageref{eqn:choice:rule}
      \\
      \(
      3\myspace
      m\myspace
      \iota(a_1)\myspace\dots\myspace\iota(a_{m})\myspace
      o\!-\!m\myspace
      o\!-\!n\myspace
      \iota(a_{n+1})\myspace\dots\myspace\iota(a_{o})\myspace
      \iota(a_{m+1})\myspace\dots\myspace\iota(a_{n})
      \)\\
      \hline
      Weight rule
      % \eqref{eqn:weight:rule},
      Slide~\pageref{eqn:weight:rule}
      \\
      \(
      5\myspace
      \iota(a_0)\myspace
      l\myspace
      n\myspace
      n\!-\!m\myspace
      \iota(a_{m+1})\myspace\dots\myspace\iota(a_{n})\myspace
      \iota(a_1)\myspace\dots\myspace\iota(a_{m})\myspace
      w_{m+1}\myspace\dots\myspace w_n\myspace
      w_1\myspace\dots\myspace w_m
      \)\\
      \hline
      Minimize statement\footnotemark
      % \eqref{eq:minimize},
      Slide~\pageref{eq:minimize}
      \\
      \(
      6\myspace
      0\myspace
      n\myspace
      n\!-\!m\myspace
      \iota(a_{m+1})\myspace\dots\myspace\iota(a_{n})\myspace
      \iota(a_1)\myspace\dots\myspace\iota(a_{m})\myspace
      w_{m+1}\myspace\dots\myspace w_n\myspace
      w_1\myspace\dots\myspace w_m
      \)\\
      \hline
      Disjunctive rule
      % \eqref{eqn:rule:disjunctive},
      Slide~\pageref{eqn:rule:disjunctive}
      \\
      \(
      8\myspace
      m\myspace
      \iota(a_1)\myspace\dots\myspace\iota(a_{m})\myspace
      o\!-\!m\myspace
      o\!-\!n\myspace
      \iota(a_{n+1})\myspace\dots\myspace\iota(a_{o})\myspace
      \iota(a_{m+1})\myspace\dots\myspace\iota(a_{n})
      \)
      \\
      \hline
    \end{tabular}
    \\
    \medskip
    \normalsize
    \begin{itemize}
    \item<1-> The function $\iota$ represents a mapping of atoms to numbers
    \end{itemize}
\end{frame}
% ----------------------------------------------------------------------
\subsection{aspif format}
% ----------------------------------------------------------------------
\begin{frame}{\aspif\ format}
  \begin{itemize}
  \item The \aspif\ format consists of
    \begin{itemize}
    \item rule statements
    \item minimize statements
    \item projection statements
    \item output statements
    \item external statements
    \item assumption statements
    \item heuristic statements
    \item edge statements
    \item theory terms and atoms
    \item comments
    \end{itemize}
  \item Line-oriented format
  \end{itemize}
\end{frame}
% ----------------------------------------------------------------------
\begin{frame}{Rule statements}
  \newcommand\Space{\textvisiblespace}
  Rule statements have the form
  \hfill \(1 \Space H \Space B\)\qquad\qquad
  \begin{itemize}
  \item<2-> Head $H$ has form
    \hfill\(h \Space m \Space a_1 \Space \dots \Space a_m\)\qquad\qquad
    \begin{itemize}
    \item $h \in \{\texttt{0},\texttt{1}\}$ determines whether\\ the head is a disjunction or choice,
    \item $m \geq 0$ is the number of head elements, and
    \item each $a_i$ is a positive literal
    \end{itemize}
  \item<2->[] Heads are disjunctions or choices, including the special case of singular disjunctions for representing normal rules.
  \item<3-> Body $B$ has one of two forms
    \begin{itemize}
    \item normal bodies have form
      \hfill\(\texttt{0} \Space n \Space l_{1} \Space \dots \Space l_n\)\qquad\qquad
      \begin{itemize}
      \item $n \geq 0$ is the length of the rule body, and
      \item each $l_i$ is a literal.
      \end{itemize}
    \item weight bodies have form
      \hfill\(\texttt{1} \Space l \Space n \Space l_1 \Space w_1  \Space \dots \Space l_n \Space w_n\)\qquad\qquad
      \begin{itemize}
      \item $l$ is a positive integer to denote the lower bound,
      \item $n \geq 0$ is the number of literals in the rule body, and
      \item each $l_i$ and $w_i$ are a literal and a positive integer
      \end{itemize}
    \end{itemize}
%  \item All types of ASP rules are included in the above rule format
  \end{itemize}
\end{frame}
% ----------------------------------------------------------------------
\begin{frame}[fragile,c]{Example}
\begin{center}
{\begin{minipage}[t]{0.3\textwidth}
\lstinputlisting[basicstyle=\ttfamily]{language/ezy.lp}%
\end{minipage}}%
\qquad\pause\qquad%\frame%
{\begin{minipage}[t]{0.3\textwidth}
\lstinputlisting[basicstyle=\ttfamily]{language/ezy.aspif}%
\end{minipage}}%
\end{center}
\end{frame}
% ----------------------------------------------------------------------
%%% Local Variables:
%%% mode: latex
%%% TeX-master: "../asp"
%%% End:
