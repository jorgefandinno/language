% ----------------------------------------------------------------------
\begin{frame}{\smodels\ format}
  \begin{itemize}
  \item The \smodels\ format consists of
    \begin{itemize}
    \item normal rules
    \item choice rules
    \item cardinality rules
    \item weight rules
    \item minimization statements
    \end{itemize}
  \item Block-oriented format
  \item<2-> \structure{Note} Minimization statements are not part of the logic program
  \end{itemize}
\end{frame}
% ----------------------------------------------------------------------
\begin{frame}[c]{\smodels\ format in detail}
  \footnotesize
  \newcommand{\myspace}{\mbox{\textvisiblespace}}
  \begin{tabular}{|l|}
      \hline
      Type/Format\\
      \hline
      \hline
      Normal rule
      % \eqref{eqn:rule},
      Slide~\pageref{eqn:rule}
      \\
      \(
      1\myspace
      \iota(a_0)\myspace
      n\myspace
      n\!-\!m\myspace
      \iota(a_{m+1})\myspace\dots\myspace\iota(a_{n})\myspace
      \iota(a_1)\myspace\dots\myspace\iota(a_{m})
      \)\\
      \hline
      Cardinality rule
      % \eqref{eqn:cardinality:rule},
      Slide~\pageref{eqn:cardinality:rule}
      \\
      \(
      2\myspace
      \iota(a_0)\myspace
      n\myspace
      n\!-\!m\myspace
      l\myspace
      \iota(a_{m+1})\myspace\dots\myspace\iota(a_{n})\myspace
      \iota(a_1)\myspace\dots\myspace\iota(a_{m})
      \)\\
      \hline
      Choice rule
      % \eqref{eqn:choice:rule},
      Slide~\pageref{eqn:choice:rule}
      \\
      \(
      3\myspace
      m\myspace
      \iota(a_1)\myspace\dots\myspace\iota(a_{m})\myspace
      o\!-\!m\myspace
      o\!-\!n\myspace
      \iota(a_{n+1})\myspace\dots\myspace\iota(a_{o})\myspace
      \iota(a_{m+1})\myspace\dots\myspace\iota(a_{n})
      \)\\
      \hline
      Weight rule
      % \eqref{eqn:weight:rule},
      Slide~\pageref{eqn:weight:rule}
      \\
      \(
      5\myspace
      \iota(a_0)\myspace
      l\myspace
      n\myspace
      n\!-\!m\myspace
      \iota(a_{m+1})\myspace\dots\myspace\iota(a_{n})\myspace
      \iota(a_1)\myspace\dots\myspace\iota(a_{m})\myspace
      w_{m+1}\myspace\dots\myspace w_n\myspace
      w_1\myspace\dots\myspace w_m
      \)\\
      \hline
      Minimize statement\footnotemark
      % \eqref{eq:minimize},
      Slide~\pageref{eq:minimize}
      \\
      \(
      6\myspace
      0\myspace
      n\myspace
      n\!-\!m\myspace
      \iota(a_{m+1})\myspace\dots\myspace\iota(a_{n})\myspace
      \iota(a_1)\myspace\dots\myspace\iota(a_{m})\myspace
      w_{m+1}\myspace\dots\myspace w_n\myspace
      w_1\myspace\dots\myspace w_m
      \)\\
      \hline
      Disjunctive rule
      % \eqref{eqn:rule:disjunctive},
      Slide~\pageref{eqn:rule:disjunctive}
      \\
      \(
      8\myspace
      m\myspace
      \iota(a_1)\myspace\dots\myspace\iota(a_{m})\myspace
      o\!-\!m\myspace
      o\!-\!n\myspace
      \iota(a_{n+1})\myspace\dots\myspace\iota(a_{o})\myspace
      \iota(a_{m+1})\myspace\dots\myspace\iota(a_{n})
      \)
      \\
      \hline
    \end{tabular}
    \\
    \medskip
    \normalsize
    \begin{itemize}
    \item<1-> The function $\iota$ represents a mapping of atoms to numbers
    \end{itemize}
\end{frame}
% ----------------------------------------------------------------------
%
%%% Local Variables:
%%% mode: latex
%%% TeX-master: "../../main"
%%% End:
