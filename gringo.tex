% ----------------------------------------------------------------------
\begin{frame}{Terms and literals}
  \begin{itemize}
  \item<2-> \structure{Terms} \ $t$ \
    \only<2-4>{are formed from
    \begin{itemize}
    \item<2-> constant\only<2>{ symbol}s, eg \texttt{c}, \texttt{d}, \dots
    \item<2-> function\only<2>{ symbol}s, eg \texttt{f}, \texttt{g}, \dots
    \item<2-> numeral\only<2>{ symbol}s, eg 1, 2, \dots
    \item<2-> variable\only<2>{ symbol}s, eg \texttt{X}, \texttt{Y}, \dots, \texttt{\_}
    \item<2-> parentheses $($, $)$
    \item<2-> tuple delimiters $\langle$, $\rangle$ \only<2>{ \ (omitted whenever possible)}
    \end{itemize}
    \only<4>{eg \ \texttt{f(3,c,Z)}, \texttt{g(42,\_,\_)}, or \texttt{f((3,c),X)}}}
  \item<5-> \structure{Tuples} \ $\boldsymbol{t}$\only<5>{ of terms}\only<15->{, $\boldsymbol{L}$}\only<15>{ of literals}
  \item<6-> \structure{\only<6>{(Negated) }Atoms} \ $a$, $-a$\only<9->{, $\bot$, $\top$}
    \only<6-8>{ are formed from
    \begin{itemize}
    \item<6-> predicate\only<6>{ symbol}s, eg \texttt{p}, \texttt{q}, \dots
    \item<6-> parentheses $($, $)$
    \item<6-> tuples of terms
    \end{itemize}
    \only<8>{eg \ \texttt{-p(f(3,c,Z),g(42,\_,\_))} or \texttt{q()} written as \texttt{q}}}
    \only<10>{\par viz \ \texttt{\#false} and \texttt{\#true}}
  \item<11-> \structure{Symbolic literals} \ $a$, $\neg a$, $\neg \neg a$
    \\
    \only<12>{eg \ \texttt{p(a,X)}, `\texttt{not~p(a,X)}', `\texttt{not~not~p(a,X)}'}
  \item<13-> \structure{Arithmetic literals} \ $t_1\prec t_2$
    \only<13-14>{\ where
      \begin{itemize}
      \item<13-> $t_1$ and $t_2$ are terms
      \item<12-> $\prec$ is a comparison symbol
      \end{itemize}
      \only<14>{eg \ \texttt{3<1} or \texttt{f(42)=X}}}
  \item<15-> \structure{Conditional literals} \ $l:\boldsymbol{L}$
    \only<15-17>{\ where
      \begin{itemize}
      \item<15-> $l$ is a symbolic or arithmetic literal
      \item<15-> $\boldsymbol{L}$ is a tuple of symbolic or arithmetic literals
      \item<16|only@16> $l:\boldsymbol{L}$ is written as $l$ whenever $\boldsymbol{L}$ is empty
      \end{itemize}
      \only<17>{eg \ `\texttt{p(X,Y):q(X),r(Y)}' or \texttt{p(42)} or `\texttt{\#false:q}'}}
  \item<18-> \structure{Aggregate atoms} \
    \(
    s_1\prec_1\alpha\{\boldsymbol{t}_1:\boldsymbol{L}_1;\dots;\boldsymbol{t}_n:\boldsymbol{L}_n\}\prec_2 s_2
    \)
    \only<18-22>{\ where
      \begin{itemize}
      \item<18-> $\alpha$ is an aggregate name
      \item<18-> $\boldsymbol{t}_1:\boldsymbol{L}_1,\dots,\boldsymbol{t_n}:\boldsymbol{L_n}$ are conditional literals
      \item<18-> $\prec_1$ and $\prec_2$ are comparison symbols
      \item<18-> $s_1$ and $s_2$ are terms
      \item<19|only@19> one (or both) of `$s_1\prec_1$' and `$\prec_2 s_2$' can be omitted
      \item<20|only@20> omitting $\prec_1$ or $\prec_2$ defaults to $\leq$
      \end{itemize}
      \only<21-22>{eg \ \texttt{10\;\only<21>{<=\;}{\#sum}\;\{6,C:course(C);\;3,S:seminar(S)\}\only<21>{\;<=}\;20}}}
  \item<23-> \structure{Aggregate literals} \ $a$, $\neg a$, $\neg \neg a$
    \only<23-24>{\ where
      \begin{itemize}
      \item<23-> $a$ is an aggregate atom
      \end{itemize}
    \only<24>{eg \ \texttt{not 10\;{\#sum}\;\{6,C:course(C);\;3,S:seminar(S)\}\;20}}}
  \item<25-> \structure{Literals} \only<20->{are conditional or aggregate literals}
    \medskip
  \item<only@26> For a detailed account please consult the user's guide!
  \end{itemize}
\end{frame}
% ----------------------------------------------------------------------
\begin{frame}{Rules}
  \begin{itemize}
  \item \structure{Rules} are of the form
    \begin{equation}
      \label{eq:aspcore:normal:rule}
       l_1\,;\dots\;;l_m\leftarrow  l_{m+1},\dots, l_n
    \end{equation}
    where
    \begin{itemize}
    \item $l_i$ is a conditional literal for $1\leq i\leq m$ and
    \item $l_i$ is a literal for $m+1\leq i\leq n$
    \end{itemize}
    \medskip
  \item<2-> \structure{Note} \ Semicolons `;' must be used in \eqref{eq:aspcore:normal:rule} instead of commas `,'
    \par whenever some $l_i$ is a (genuine) conditional literal for $1\leq i\leq n$
  \item<3-> \structure{Example} \  \texttt{a(X) :- b(X) : c(X), d(X)\alert{;} e(x).}
  \item<4-> \structure{Note} \ $l_1\,;\dots\;;l_m\leftarrow  l_{m+1},\dots, l_n$ is the same as
  \\ \phantom{Note} \ $l_1\,;\dots\;;l_m\leftarrow  l_{m+1};\dots;l_n$
  \end{itemize}
\end{frame}
% ----------------------------------------------------------------------
\begin{frame}{Shortcuts}
  \begin{itemize}
  \item<2-> A rule of the form
    \[
    \alert<1>{s_1\prec_1\alpha\{\boldsymbol{t}_1:l_1:\boldsymbol{L}_1;\dots;\boldsymbol{t}_k:l_k:\boldsymbol{L}_k\}\prec_2 s_2}
    \ \leftarrow \
     l_{m+1},\dots, l_n
    \]
    where
    \begin{itemize}
    \item $\alpha$, $\prec_i$, $s_i$, $\boldsymbol{t}_j$ are as given above for $i=1,2$ and $1\leq j\leq k$
    \item $l_j:\boldsymbol{L}_j$ is a conditional literal for $1\leq j\leq k$
    \item $l_i$ is a literal for $m+1\leq i\leq n$ \ (as in \eqref{eq:aspcore:normal:rule})
    \end{itemize}
    \medskip
  \item<3->[] is a shorthand for the following $k+1$ rules
    \begin{align*}\hspace{-10pt}
      \{ l_j\}&\leftarrow  l_{m+1},\dots, l_n,\boldsymbol{L}_j\qquad\qquad\text{ for }1\leq j\leq k\\
                &\leftarrow  l_{m+1},\dots, l_n,\neg{~s_1\prec_1\alpha\{\boldsymbol{t}_1:l_1,\boldsymbol{L}_1;\dots;\boldsymbol{t}_k:l_k,\boldsymbol{L}_k\}\prec_2 s_2}
      \end{align*}
  \item<3-> \structure{Example} \
    \texttt{10 < {\#sum} \{ C,X,Y : edge(X,Y), cost(X,Y,C) \}.}
  \end{itemize}
\end{frame}
% ----------------------------------------------------------------------
\begin{frame}{Shortcuts}
  \begin{itemize}
  \item The expression
    \[
    s_1~\{ l_1:\boldsymbol{L}_1;\dots;l_k:\boldsymbol{L}_k\}~s_2
    \]
    is a shortcut for

    \begin{itemize}\normalsize
    \item
      \(
      s_1\leq\mathit{count}\{t_1:l_1:\boldsymbol{L}_1;\dots;t_k:l_k:\boldsymbol{L}_k\}\leq s_2
      \)
      \\\smallskip
      if it appears in the head of a rule and
      \\\medskip
    \item
      \(
      s_1\leq\mathit{count}\{t_1:l_1,\boldsymbol{L}_1;\dots;t_k:l_k,\boldsymbol{L}_k\}\leq s_2
      \)
      \\\smallskip
      if it appears in the body of a rule
    \end{itemize}
    where $t_i\neq t_j$ whenever $l_i\neq l_j$ for $i\neq j$ and $1\leq i,j\leq k$
    \medskip
  \item<2-> \structure{Note} \ one (or both) of $s_1$ and $s_2$ can be omitted
  \end{itemize}
\end{frame}
% ----------------------------------------------------------------------
\begin{frame}[fragile]{Aggregates on sets}
    \medskip
  \begin{itemize}
  \item<1-> \structure{Example} \
    \texttt{sum(S) :- S = {\#sum} \{ C\only<1-2,12->{,X,Y} : travel(X,Y,C) \}.}
  \item<2->[] Assume
    `\texttt{travel(a,b,3).} \texttt{travel(b,c,3).}'
  \item<4-11,13->[] We get \ \texttt{sum(\alt<13->{6}{3})}\,!
    \only<-19>{\bigskip}
  \item<only@5-11,14-19> \structure{How come?}
  \item<only@6-11,14->[]
    {\tt
      \only<10-11,18-19>{sum(\alt<18-19>{6}{3})}\only<10,18>{ :- \alt<18>{6}{3} = }%
      \only<6-10,14-18>{{\#sum} \{
        \alt<9-11,17->{\alt<17->{(3,a,b); (3,b,c)}{3}}%
                  {\alt<7-8,15-16>{3\only<15-16>{,a,b}\only<7,15>{\;:\;travel(a,b,3)}; 3\only<15-16>{,b,c}\only<7,15>{\;:\;travel(b,c,3)}}
                                  {C\only<14->{,X,Y}\;:\;travel(X,Y,C)}}}%
    \only<6-10,14-18>{ \}}%
    \only<10-11,18-19>{.}}

  \item<only@21-> \structure{Example} \
\smallskip
\begin{semiverbatim}
\uncover<2->{$ gringo --text <(echo "\{a;b\}.")}
\uncover<3->{#count\{ 1,0,a : a; 1,0,b : b \}.}
\end{semiverbatim}
\smallskip
\item<only@22->[]\itarrow\ \ \gringo\ generates two distinct term tuples \lstinline{(1,0,a)} and \lstinline{(1,0,b)}
\medskip
\item<only@23-> \structure{Example} \ Why ``;''?
  \pause[24]
\begin{lstlisting}[basicstyle=\ttfamily\small]
  { q(X,Y): p(X), p(Y), X < Y; q(X,X): p(X) } = 1
\end{lstlisting}
  \end{itemize}
\end{frame}
% ----------------------------------------------------------------------
\begin{frame}[fragile]{Weak constraints}
  \begin{itemize}
  \item \structure{Syntax} A \alert{weak constraint} is of the form
    \[
    \leftsquigarrow  l_1,\dots, l_n\ [ w@p,t_1, \dots, t_m ]
    \]
    where
    \begin{itemize}\normalsize
    \item $l_1, \dots, l_n$ are literals
    \item $t_1, \dots, t_m$, $w$, and $p$ are terms
    \end{itemize}
  \item[] $w$ and $p$ stand for a weight and priority level
    \\
    ($p=0$ if `$@p$' is omitted)
    \smallskip
  \item<2-> \structure{Example}
    The weak constraint
\begin{lstlisting}
:~ hd(I,P,C). [C@2,I]
\end{lstlisting}
    \pause[3] amounts to the minimize statement
\begin{lstlisting}
#minimize{ C@2,I : hd(I,P,C) }.
\end{lstlisting}
  \end{itemize}
\end{frame}
% ----------------------------------------------------------------------
\begin{frame}{Some more directives}
  \begin{itemize}
  \item<1-> \structure{Output}
    \[
      \#\mathtt{show}.\qquad \#\mathtt{show} \ p/n.\qquad \#\mathtt{show} \ t\; :\; l_1,\dots,l_n.
    \]
    \smallskip
  \item<2-> \structure{Projection}
    \[
      \#\mathtt{project} \ p/n.\qquad \#\mathtt{project} \ a\; :\; l_1,\dots,l_n.
    \]
    \smallskip
  \item<3-> \structure{Heuristics}
    \[
      \#\mathtt{heuristic} \ a\ :\; l_1,\dots,l_n. \ [k@p,m]
    \]
    \smallskip
  \item<4-> \structure{Acyclicity}
    \[
      \#\mathtt{edge}\; (u,v)\; :\; l_1,\dots,l_n.
    \]
  \end{itemize}
\end{frame}
% ----------------------------------------------------------------------
%
%%% Local Variables:
%%% mode: latex
%%% TeX-master: "../../main"
%%% End:
