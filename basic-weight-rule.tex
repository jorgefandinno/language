% ----------------------------------------------------------------------
\begin{frame}[fragile]{Weight rule}
  \label{eqn:weight:rule}
  \begin{itemize}
  \item<only@1-2> \structure{Purpose} \ Bound (lower) sum of subsets of literal weights
  \item \structure{Syntax} \ A \alert{weighted literal} $w:k$ associates weight $w$ with literal $k$
  \item \structure{Syntax} \ A \alert{weight rule} has the form
    \[
      a_0\leftarrow l~\{\ w_1:a_1,\dots,w_m:a_m,w_{m+1}:\neg a_{m+1},\dots,w_n:\neg a_n\ \}
    \]
    where $0\leq m\leq n$ and each $a_i$ is an atom
  \item [] and $l$ and $w_i$ are integers for $1\leq i\leq n$
    \smallskip
  \item<only@2> \structure{Informal meaning} \
    The head belongs to the stable model,
    if the sum of weights associated with positive/negative body literals in/excluded in the stable model
    is at least $l$

  \item<only@3-> \structure{Note}
    \begin{itemize}\normalsize
    \item A cardinality rule is a weight rule where $w_i=1$  for $0\leq i\leq n$
    \item<only@4-> Weight constraints generalize cardinality constraints accordingly
    \item<only@4->[] and amount to constraints on count and sum aggregate functions
    \end{itemize}

  \item<only@5-> \structure{Example}
\begin{lstlisting}[basicstyle=\ttfamily\small]
5 { 4:course(db); 6:course(ai); 3:course(xml) } 10
\end{lstlisting}
  \end{itemize}
\end{frame}
% --------------------------------------------------------------------------------
%
%%% Local Variables:
%%% mode: latex
%%% TeX-master: "../../main"
%%% End:
