% ----------------------------------------------------------------------
\begin{frame}[fragile]{Cardinality rule}
  \label{eqn:cardinality:rule}
  \begin{itemize}
  \item \structure{Purpose} \ Control (lower) cardinality of subsets of literals
    \smallskip
  \item \structure{Syntax} \
    A \alert{cardinality rule} is the form
    \[
    a_0\leftarrow l~\{\ a_1,\dots,a_m,\neg a_{m+1},\dots,\neg a_n\ \}
    \]
    where $0\leq m\leq n$ and each $a_i$ is an atom for $1\leq i\leq n$
  \item [] and $l$ is a non-negative integer called \alert{lower bound}
    \smallskip
  \item<2-> \structure{Informal meaning} \
    The head atom belongs to the stable model,\\
    if at least $l$ positive/negative body literals are in/excluded in the stable model
    \medskip
  \item<only@3> \structure{Example}
\begin{lstlisting}[basicstyle=\ttfamily\small]
pass(c42) :- 2 { pass(a1); pass(a2); pass(a3) }.
\end{lstlisting}
  \item<only@4-> \structure{Example program}
    \begin{align*}
      & \{\;a\leftarrow 1\;\{b,c\},\ b\leftarrow\;\}&\uncover<5->{\{a,b\}}
    \end{align*}
  \end{itemize}
\end{frame}
% ----------------------------------------------------------------------
\begin{frame}{Embedding in normal rules}
\begin{itemize}
\item A cardinality rule of form
\[
\alert<2>{a_0}\leftarrow\alert<2>{l}~\{\ \alert<5>{a_1,\dots,a_m},\alert<6>{\neg a_{m+1},\dots,\neg a_n}\ \}
\]
is translated into the normal rule \
\(
\alert<2>{a_0}\leftarrow \mathit{ctr}(1,\alert<2>{l})
\)
\ \only<4->{and}
\item<only@4-> [] for $0\leq k\leq l$ the rules
\[
\begin{array}{rclp{20pt}l}
\mathit{ctr}(i, k{+}1)&\leftarrow&\mathit{ctr}({i+1}, k),\alert<5>{a_i}
&&\\
\mathit{ctr}(i, k)&\leftarrow&\mathit{ctr}({i+1}, k)
&&\text{for }\alert<5>{1}\leq i\leq \alert<5>{m}
\\[5pt]
\mathit{ctr}(j, k{+}1)&\leftarrow&\mathit{ctr}({j+1}, k),\alert<6>{\neg a_j}
\\
\mathit{ctr}(j, k)&\leftarrow&\mathit{ctr}({j+1}, k)
&&\text{for }\alert<6>{m+1}\leq j\leq \alert<6>{n}
\\[5pt]
\mathit{ctr}(\alert<7>{n+1}, \alert<7>{0})&\leftarrow&
&&
\end{array}
\]
\item<3->
The atom $\mathit{ctr}(i,j)$ represents that at least $j$ of the
literals having an equal or greater index than $i$ are in a stable model
\end{itemize}
\end{frame}
% ----------------------------------------------------------------------
\begin{frame}{An example}
\begin{itemize}
\item Program
  \(
  \{  a\leftarrow,\ c\leftarrow 1~\{a,b\}  \}
  \)
  has the stable model $\{a,c\}$
\item<2-> Translating the cardinality rule yields the rules
\[
\begin{array}[t]{rcl}
a&\leftarrow&
\end{array}
\qquad
\begin{array}[t]{rcl}
c&\leftarrow& \mathit{ctr}(1,1)
\\
\mathit{ctr}(1, 2)&\leftarrow&\mathit{ctr}(2, 1),a
\\
\mathit{ctr}(1, 1)&\leftarrow&\mathit{ctr}(2, 1)
\\
\mathit{ctr}(2, 2)&\leftarrow&\mathit{ctr}(3, 1),b
\\
\mathit{ctr}(2, 1)&\leftarrow&\mathit{ctr}(3, 1)
\\
\mathit{ctr}(1, 1)&\leftarrow&\mathit{ctr}(2, 0),a
\\
\mathit{ctr}(1, 0)&\leftarrow&\mathit{ctr}(2, 0)
\\
\mathit{ctr}(2, 1)&\leftarrow&\mathit{ctr}(3, 0),b
\\
\mathit{ctr}(2, 0)&\leftarrow&\mathit{ctr}(3, 0)
\\
\mathit{ctr}(3, 0)&\leftarrow&
\end{array}
\]
having stable model
\(
\{
a,
\mathit{ctr}(3, 0),
\mathit{ctr}(2, 0),
\mathit{ctr}(1, 0),
\mathit{ctr}(1, 1),
c
\}
\)
\end{itemize}
\end{frame}
% ----------------------------------------------------------------------
\begin{frame}{\dots\ and vice versa}
  \begin{itemize}
  \item A normal rule
    \[
    a_0\leftarrow a_1,\dots,a_m, \neg a_{m+1},\dots,\neg a_n
    \]
  \item[] can be represented by the cardinality rule
    \[
    a_0\leftarrow n \ \{a_1,\dots,a_m, \neg a_{m+1},\dots,\neg a_n\}
    \]
  \end{itemize}
\end{frame}
% ----------------------------------------------------------------------
\begin{frame}{Cardinality rules with upper bounds}
  \begin{itemize}
  \item A rule of the form
    \begin{equation}\label{eq:cardinality:constraint}
      a_0\leftarrow \alert<3>{l~\{\ a_1,\dots,a_m,\neg a_{m+1},\dots,\neg a_n\ \}~u}
    \end{equation}
    where $0\leq m\leq n$ and each $a_i$ is an atom for $1\leq i\leq n$
  \item [] and $l$ and $u$ are non-negative integers
    \only<3>{\medskip}
  \item<only@2> []
    stands for
    \[
    \begin{array}{rcl}
      a_0 &\leftarrow& x, \neg y\\
      x   &\leftarrow& l~\{\ a_1,\dots,a_m,\neg a_{m+1},\dots,\neg a_n\ \}\\
      y   &\leftarrow& u{+}1~\{\ a_1,\dots,a_m,\neg a_{m+1},\dots,\neg a_n\ \}
    \end{array}
    \]
    where $x$ and $y$ are new symbols
  \item<only@3> \structure{Note} \
    The expression in the body of the cardinality rule \eqref{eq:cardinality:constraint}

    is referred to as a \alert<3>{cardinality constraint} with lower and upper

    bound $l$ and $u$
  \end{itemize}
\end{frame}
% ----------------------------------------------------------------------
\begin{frame}[fragile]{Cardinality constraints as heads}
  \begin{itemize}
  \item A rule of the form
    \[
    l~\{a_1,\dots,a_m,\neg a_{m+1},\dots,\neg a_n\}~u\leftarrow a_{n+1},\dots,a_o,\neg a_{o+1},\dots,\neg a_p
    \]
    where $0\leq m\leq n\leq o\leq p$ and each $a_i$ is an atom for $1\leq i\leq p$
  \item [] and $l$ and $u$ are non-negative integers
    \only<2>{\medskip}
  \item<only@2> \structure{Example}
\begin{lstlisting}[basicstyle=\ttfamily\small]
1 {color(2,red); color(2,green); color(2,blue)} 1.
\end{lstlisting}
  \item<only@3> []
    stands for
    \[
      \begin{array}{rcl}
        x                 &\leftarrow& a_{n+1},\dots,a_o, \neg a_{o+1},\dots,\neg a_p\\
        \{a_1,\dots,a_m\} &\leftarrow& x\\
        y                 &\leftarrow& l~\{a_1,\dots,a_m,,\neg a_{m+1},\dots,\neg a_n\}~u\\
                          &\leftarrow& x, \neg y
      \end{array}
    \]
    where $x$ and $y$ are new symbols
  \end{itemize}
\end{frame}
% ----------------------------------------------------------------------
\begin{frame}{Full-fledged cardinality rules}
  \begin{itemize}
  \item A rule of the form
    \[
      \alert<6>{l_0~S_0~u_0}\leftarrow\alert<4>{l_1~S_1~u_1},\dots,\alert<4>{l_n~S_n~u_n}
    \]
    where each $\alert<3>{l_i~S_i~u_i}$ is a cardinality constraint for $0\leq i\leq n$
  \item<only@2-> [] stands for % $0\leq i\leq n$
    \[
      \begin{array}{rclp{20pt}rcl}
                    \alert<5>{a}&\alert<5>{\leftarrow}&\multicolumn{5}{l}{\alert<4,5>{b_1},\dots,\alert<4,5>{b_n},\alert<4,5>{\neg c_1},\dots,\alert<4,5>{\neg c_n}}\\[10pt]
        \alert<6>{\poslits{S_0}}&\alert<6>{\leftarrow}&\alert<5>{a}\\
                                &\alert<6>{\leftarrow}&\alert<5>{a}\alert<6>{,\neg b_0}&&\alert<3-4,6>{b_i}&\alert<3-4,6>{\leftarrow}&\alert<3-4,6>{l_i~S_i}\\
                                &\alert<6>{\leftarrow}&\alert<5>{a}\alert<6>{,     c_0}&&\alert<3-4,6>{c_i}&\alert<3-4,6>{\leftarrow}&\alert<3-4,6>{u_i{+}1~S_i}
      \end{array}
    \]
    where $a,b_i,c_i$ are new symbols and $S_0^+$ gives all atoms in $S_0$
  \end{itemize}
\end{frame}
% ----------------------------------------------------------------------
%
%%% Local Variables:
%%% mode: latex
%%% TeX-master: "../../main"
%%% End:
